\documentclass[12pt,parskip=full,chapterprefix=true,a5paper]{scrbook}
\usepackage{amsmath}
\usepackage{amsthm}
\usepackage{fontspec}
\setmainfont{TeX Gyre Pagella}
\setsansfont[BoldFont={Source Sans Pro SemiBold}]{Source Sans Pro Light}
\usepackage[math-style=french]{unicode-math}
\setmathfont{TeX Gyre Pagella Math}
\usepackage{tikz}
\usepackage{enumitem}
\usepackage[french]{babel}
\usepackage{microtype}
\usepackage{hyperref}

\hypersetup{%
  hidelinks=true,
  unicode=true,
  pdftitle={Cours élémentaire de mathématiques supérieures},
  pdfauthor={Jean Quinet},
  pdfcreator={Vincent Garibal},
  pdfsubject={Mathématiques},
  pdfkeywords={Algèbre, Analyse, Géométrie},
}

\usetikzlibrary{patterns}

\title{Cours élémentaire de mathématiques supérieures}
\author{Jean Quinet}

\begin{document}

\frontmatter

\maketitle
\tableofcontents

\mainmatter

\part{Algèbre}

\chapter{Les ensembles}

\section{Ensembles et relations}
Le langage des ensembles, introduit il y a plus d'un siècle par Georg
\textsc{Cantor} (1845-1918), a envahi toutes les branches des mathématiques ; il
est employé en France à tous les niveaux de l'enseignement, y compris
à l'école maternelle. Nous employons donc ce langage, non seulement
parce qu'il est commode, mais aussi et surtout parce qu'il est est
désormais impossible de l'ignorer.

Les notions d'\emph{ensemble} et de \emph{relation} sont premières,
c'est à dire qu'il n'est pas possible de les définir à partir d'autres
notions.

Intuitivement, une relation est une affirmation portant sur un
ensemble \(E\), qui est vérifiée ou non pour un ensemble donné \(E_0\)
; suivant le cas, on dit que \(E_0\) satisfait ou non à cette
relation.

\subsection*{Négation d'une relation}
Étant donnée une relation \(R\), on définit en logique une nouvelle
relation, appelée \emph{négation} de \(R\), et notée
\((\text{non }R)\), ou encore \(\neg R\). La négation de la négation
de \(R\) est la relation \(R\) elle-même.

On dit encore que les relations \(R\) et \(\neg R\) sont
contradictoires, c'est à dire que
\begin{itemize}
\item pour tout ensemble, l'une des deux relations est vraie,
\item mais pour aucun ensemble, elles ne sont vraies l'une et l'autre.
\end{itemize}

\medskip Par exemple, dans l'ensemble des entiers naturels, les
relations
\begin{flalign*}
     & R\text{ : l'entier }x\text{ est pair,}\\
\neg & R\text{ : l'entier }x\text{ n'est pas pair (autrement dit,
  est impair)}
\end{flalign*}
sont contradictoires.

\subsection*{Conjonction, disjonction}
On appelle \emph{conjonction} de deux relations \(R_1\) et \(R_2\), et on note (\(R_1\) et \(R_2\)), ou \(R_1\wedge R_2\), la relation vraie si et seulement si \(R_1\) et \(R_2\) le sont. On définit enfin la \emph{disjonction} des relations \(R_1\) et \(R_2\) : c'est la relation, notée (\(R_1\) ou \(R_2\)), ou \(R_1\vee R_2\), vraie si et seulement si l'une au moins des deux relations \(R_1\) et \(R_2\) est vraie.

\subsection*{Implication}
À partir de ces trois définitions fondamentales, on introduit l'\emph{implication} : étant données deux relations \(R_1\) et \(R_2\), la relation (\(R_2\) ou (non \(R_1\)) se note \(R_1\Rightarrow R_2\), et se lit «~\(R_1\) implique \(R_2\)~». Si l'implication \(R_1\Rightarrow R_2\) est vraie, tout ensemble satisfaisant à \(R_1\) satisfait à \(R_2\).

La relation d'implication conduit à la notion d'\emph{équivalence} de deux relations : on dit que les relations \(R_1\) et \(R_2\) sont équivalentes, et on note \(R_1\Leftrightarrow R_2\), si chacune d'elles implique l'autre.

L'implication \(R_2\Rightarrow R_1\) est dite \emph{réciproque} de l'implication \(R_1\Rightarrow R_2\). L'implication \((\text{non}R_2)\Rightarrow(\text{non} R_1)\) est appelée \emph{contraposée} de \(R_1\Rightarrow R_2\). Une implication et sa contraposée sont toujours équivalentes.

\paragraph{Exemple.} Soient les relations
\begin{flalign*}
  R_1& \text{ : le triangle ABC a ses trois côtés égaux,}\\
  R_2& \text{ : le triangle ABC a ses trois angles égaux.}
\end{flalign*}
On a l'implication \(R_1\Rightarrow R_2\) (c'est le théorème bien connu : si un triangle a ses trois côtés égaux, il a aussi ses trois angles égaux).

On a aussi l'implication réciproque \(R_2\Rightarrow R_1\) (c'est le théorème : si un triangle a ses trois angles égaux, il a aussi ses trois côtés égaux).

Les deux relations sont donc \emph{équivalentes} :
\[R_1\Leftrightarrow R_2.\]

\section{Égalité. Appartenance}
Les relations d'égalité et d'appartenance sont encore des notions premières.

L'\emph{égalité} de deux ensembles \(E\) et \(F\) se note \(E=F\) (et se lit «~\(E\) égale \(F\)~») ; elle signifie intuitivement que les lettres \(E\) et \(F\) représentent le même objet.

Pour tout ensemble \(E\), \(E=E\). La relation \(E=F\) équivaut à la relation \(F=E\). Enfin, si \(E=F\) et si \(F=G\), alors \(E=G\). La négation de l'égalité de \(E\) et de \(F\) se note \(E\neq F\) (et se lit «~\(E\) n'égale pas \(F\)~», ou ~«\(E\) est différent de \(F\)~»).

L'\emph{appartenance} d'un ensemble \(x\) à un ensemble \(E\) se note \(x\in E\) (et se lit «~\(x\) appartient à \(E\)~»). On dit alors que \(x\) est \emph{élément} de \(E\). La négation de l'appartenance de \(x\) à \(E\) se note \(x\notin E\) (et se lit bien entendu «~\(x\) n'appartient pas à \(E\)~»).

Pour que deux ensembles \(E\) et \(F\) soient égaux, il faut et il suffit que les relations d'appartenance à \(E\) et à \(F\) soient équivalentes :
\[
  x\in E \Leftrightarrow x\in F
\]
En particulier, l'énoncé précédent affirme l'unicité d'une ensemble n'ayant aucun élément. On admet l'existence d'un tel ensemble, appelé \emph{ensemble vide}, et noté \(\varnothing\).

On dit qu'un ensemble \(E\) a un élément s'il est non vide et si la conjonction des relations \(x\in E\) et \(y\in E\) implique \(x=y\). On note un tel ensemble \(\{x\}\), et on l'appelle \emph{singleton}. On définit de même les ensembles à deux éléments, appelés \emph{paires} ; on emploie alors la notation \(\{x,y\}\).

Plus généralement, on peut définir un ensemble \(E\) par la liste de ses éléments : \(E=\{x,y,z,\dots\}\) ; on dit que l'ensemble \(E\) est donné en \emph{extension}. Dans certains cas, on peut aussi définir un ensemble \(E\) par une relation caractéristique \(R\) ; l'ensemble \(E\) est alors donné en \emph{compréhension}.

On représente généralement un ensemble par la partie du plan limitée par un contour fermé (Fig. \ref{fig:ensemble}).

\begin{figure}[ht]
  \centering
  \begin{tikzpicture}
    \draw (1,0) circle [x radius=2cm, y radius=12mm, rotate=30];
    \node[fill=white] at (1.5cm,1.4cm) {E};
  \end{tikzpicture}
  \caption{\label{fig:ensemble}Ensemble.}
\end{figure}

\section{Parties d'un ensemble}
Soient \(E\) et \(F\) deux ensembles. On dit que \(E\) est inclus (ou encore contenu) dans \(F\) si tout élément de \(E\) est élément de \(F\), ce qu'on note \(E\subset F\). On dit encore que \(E\) est une partie (ou un sous-ensemble) de \(F\) (Fig. \ref{fig:inclusion}).

\begin{figure}[ht]
  \centering
  \begin{tikzpicture}
    \draw (0,0) circle [x radius=2cm, y radius=1.5cm, rotate=-20];
    \node[fill=white] at (1,1.1) {F};
    \draw (0.5,-0.3) circle [x radius=1cm, y radius=0.7cm, rotate=10];
    \node[fill=white] at (-0.2,0) {E};
  \end{tikzpicture}
  \caption{\label{fig:inclusion}Inclusion.}
\end{figure}

\paragraph{Exemples}
\begin{enumerate}
\item L'ensemble vide est inclus dans tout ensemble.
\item L'ensemble des nombres entiers pairs est une partie de l'ensemble des nombres entiers naturels.
\end{enumerate}

La relation d'inclusion possède évidemment les propriétés suivantes :
\begin{enumerate}[label=\alph*)]
\item Tout ensemble est contenu dans lui-même.
\item Si un ensemble \(E\) est inclus dans un ensemble \(F\), et si \(F\) est inclus dans \(E\), alors \(E\) est égal à \(F\).
\item Soient \(E\), \(F\) et \(G\) trois ensembles. Si \(E\subset F\) et si \(F\subset G\), alors \(E\subset G\).
\end{enumerate}

Les parties d'un ensemble \(E\) constituent un ensemble, appelé naturellement \emph{ensemble des parties de \(E\)}, et noté \(\mathcal{P}(E)\).

\paragraph{Exemples}
\begin{enumerate}
\item Pour tout ensemble \(E\), la partie vide de \(E\); et l'ensemble \(E\) lui-même, appartiennent à l'ensemble \(\mathcal{P}(E)\).
\item Lorsque \(E=\varnothing\), l'ensemble \(\mathcal{P}(E)\) a un seul élément, à savoir la partie vide de \(E\) :
  \[
    \mathcal{P}(\varnothing)=\{\varnothing\}.
  \]
\end{enumerate}

\subsection*{Complémentaire d'une partie d'un ensemble}
Soient \(E\) un ensemble, et \(P\) une partie de \(E\). L'ensemble des éléments de \(E\) qui n'appartiennent pas à \(P\) s'appelle complémentaire de \(P\) dans \(E\), et se note \(\complement_EP\), ou plus simplement \(\overline{P}\), lorsque aucune confusion n'est à craindre (Fig. \ref{fig:complementaire}).

On emploie la notation \(E-P\), qui sera généralisée plus loin.

\begin{figure}[ht]
  \centering
  \begin{tikzpicture}
    \draw[pattern=dots] (0,0) circle [x radius=2cm, y radius=1.5cm, rotate=-20];
    \node[circle,inner sep=1pt,fill=white] at (1,1.1) {E};
    \draw[fill=white] (0.5,-0.3) circle [x radius=1cm, y radius=0.7cm, rotate=10];
    \node[circle,inner sep=1pt,fill=white] at (-0.2,0) {P};
    \node at (-0.5,0.8) {\(\overline{P}\)};
  \end{tikzpicture}
  \caption{\label{fig:complementaire}Complémentaire d'une partie.}
\end{figure}

Le passage aux complémentairee possède évidemment les propriétés suivantes :
\begin{enumerate} [label=\alph*)]
\item Le complémentaire dans \(E\) du complémentaire \(\overline{P}\) d'une partie \(P\) de \(E\) n'est autre que la partie \(P\) elle-même :
  \[
    \overline{P}=P
  \]
  C'est pourquoi on dit souvent que les parties \(P\) et \(\overline{P}\) sont complémentaires dans \(E\).
\item Soient \(P\) et \(Q\) deux parties de \(E\). La relation \(P=Q\) a lieu si et seulement si \(\overline{Q}=\overline{P}\) ; la relation \(P\subset Q\) a lieu si et seulement si \(\overline{Q}\subset\overline{P}\).
\end{enumerate}

\paragraph{Exemples.}
\begin{enumerate}
\item Le complémentaire dans \(E\) de la partie vide de \(E\) est égale à \(E\) tout entier :
    \[
      \complement_E\varnothing=E.
    \]
    De même que le complémentaire de la partie de \(E\) égale à \(E\) tout entier n'est autre que la partie vide de \(E\) :
    \[
      \complement_EE=\varnothing.
    \]
\item Si \(E=\{a,b,c,d\}\) et \(P=\{a,c\}\) :
    \[
      \complement_EP=\{b,d\}.
    \]
\item Le complémentaire dans l'ensemble \(\mathbb{N}\) des entiers naturels du sous-ensemble \(P\) des entiers impairs est l'ensemble des entiers pairs.
\end{enumerate}

\subsection*{Partitions}
Soient \(E\) un ensemble, et \(\mathcal{P}(E)\) l'ensemble des ses parties. On appelle \emph{partition} de \(E\) toute partie \(Q\) de \(\mathcal{P}(E)\) constituée de parties non vides de \(E\), telle que tout élément de \(E\) appartienne à un élément et un seul de \(Q\).

\paragraph{Exemple.}
Une partie de \(\mathcal{P}(E)\) constituée d'une partie \(P\) de \(E\), non vide et distincte de \(E\), et de la partie complémentaire \(\overline{P}\), est une partition de E. Ainsi, l'ensemble des nombres pairs et l'ensemble des nombres impairs constituent une partition de \(\mathbb{N}\).

\section{Opérations sur les ensembles}

\subsection*{Intersection de deux ensembles}
Soient \(E\) et \(F\) deu ensembles. On appelle intersection de \(E\) et de \(F\) l'ensemble, noté \(E\cap F\), constitué des éléments appartenant à la fois à \(E\) et à \(F\) (Fig. \ref{fig:intersection}). Le symbole \(\cap\) se lit «~inter~».

\begin{figure}[ht]
  \centering
  \begin{tikzpicture}
    \begin{scope}
      \draw (1.5,0) circle [x radius=1cm,y radius=2cm, rotate=30];
      \draw[clip] (0,0) circle [x radius=2cm,y radius=1cm];
      \fill[pattern=dots] (1.5,0) circle [x radius=1cm,y radius=2cm,rotate=30];
    \end{scope}
    \node[circle,inner sep=1pt,fill=white] at (-1,0.9) {E};
    \node[circle,inner sep=1pt,fill=white] at (1.9,1.1) {F};
    \node[inner sep=3pt,fill=white] at (1.1,0.1) {\(E\cap F\)};
  \end{tikzpicture}
  \caption{\label{fig:intersection}Intersection de deux ensembles.}
\end{figure}

Lorsque l'intersection de \(E\) et de \(F\) est l'ensemble vide on dit que les ensembles \(E\) et \(F\) sont disjoints (Fig. \ref{fig:disjoints}).

\begin{figure}[ht]
  \centering
  \begin{tikzpicture}
    \draw (2,0) circle [x radius=1.2cm,y radius=1.8cm, rotate=30];
    \draw (-1,0) circle [x radius=1.5cm,y radius=0.8cm, rotate=20];
    \node[circle,inner sep=1pt,fill=white] at (-1,0.8) {E};
    \node[circle,inner sep=1pt,fill=white] at (1.9,1.6) {F};
  \end{tikzpicture}
  \caption{\label{fig:disjoints}Ensembles disjoints.}
\end{figure}

\clearpage

\paragraph{Exemples}
\begin{enumerate}
\item L'intersection de l'ensemble des entiers rationnels multiples de 2 et de l'ensemble des entiers rationnels multiples de 3 est l'ensemble des multiples de 6.
\item L'intersection d'une partie \(P\) d'un ensemble \(E\) et de son complémentaire dans \(E\) est vide :
  \[
    P\cap\overline{P}=\varnothing
  \]
\item Plus généralement, deux parties d'un ensemble \(E\) appartenant à une même portion de \(E\) sont disjoints.
\end{enumerate}

Les propritétés suivantes sont immédiates :
\begin{enumerate} [label=\alph*)]
\item quel que soit l'ensemble \(E\),
  \[E\cap E=E;\]
\item quels que soient les ensembles \(E\) eet \(F\),
  \[E\cap F=F\cap E;\]
\item quels que soient les ensembles \(E\), \(F\) et \(G\),
  \[(E\cap F)\cap G=E\cap(F\cap G).\]
\end{enumerate}
Les parenthèses étant désormais inutiles, nous noterons \(E\cap F\cap G\) la valeur commune des deux membres (Fig. \ref{fig:intersection3ensembles}).

\begin{figure}[ht]
  \centering
  \begin{tikzpicture}
    \draw (0,0) circle [x radius=2cm,y radius=4cm];
    \draw (0,0) circle [x radius=2cm,y radius=4cm,rotate=120];
    \draw (0,0) circle [x radius=2cm,y radius=4cm,rotate=240];
    \begin{scope}
      \clip (0,0) circle [x radius=2cm,y radius=4cm];
      \clip (0,0) circle [x radius=2cm,y radius=4cm, rotate=120];
      \fill[pattern=dots] (0,0) circle [x radius=2cm, y radius=4cm,rotate=240];
    \end{scope}
    \node[circle,inner sep=1pt,fill=white] at (0,4) {E};
    \node[circle,inner sep=1pt,fill=white] at (-3.4,2) {G};
    \node[circle,inner sep=1pt,fill=white] at (3.4,2) {F};
    \node[inner sep=3pt,fill=white] at (0,0) {\(E\cap F\cap G\)};
  \end{tikzpicture}
  \caption{\label{fig:intersection3ensembles}Intersection de trois ensembles.}
\end{figure}

\subsection*{Réunion de deux ensembles}

Soient \(E\) et \(F\) deux ensembles. En appelle réunion de \(E\) et de \(F\) l'ensemble, noté \(E\cup F\), constitué des éléments appartenant à l'un au moins des ensembles \(E\) et \(F\) (Fig. \ref{fig:reunion}). Le symbole \(\cup\) se lit «~union~».

\begin{figure}[ht]
  \centering
  \begin{tikzpicture}
    \draw[pattern=dots] (0,0) circle [x radius=2cm,y radius=4cm,rotate=30];
    \draw[pattern=dots] (2,0) circle [x radius=1cm,y radius=2cm,rotate=120];
    \node[circle,inner sep=1pt,fill=white] at (-1.1,3.6) {E};
    \node[circle,inner sep=1pt,fill=white] at (3,1.4) {F};
  \end{tikzpicture}
  \caption{\label{fig:reunion}Réunion de deux ensembles.}
\end{figure}

On pourrait dire plus brièvement que la réunion est l'ensemble des éléments appartenant à \(E\) ou à \(F\) ; mais le mote «~ou~» a deux acceptions :
\begin{description}
\item [OU inclusif,]signifiant qu'un élément appartenant à \(E\) ou a \(F\) peut appartenir à ces deux ensembles à la fois ;
\item [OU exclusif,]signifiant qu'un élément appartenant à \(E\) ou à \(F\) appartient soit à \(E\), soit à \(F\), mais non à \(F\cap F\).
\end{description}
Le OU inclusif traduit la notion de réunion. Le OU exclusif conduit à la notion de différence symétrique (voir ci-dessous).

\paragraph{exemples}
\begin{enumerate}
\item La réunion de l'ensemble des nombres entiers rationnels multiples de 4 et de l'ensemble des nombres entiers rationnels de la forme \(4p+2\), où \(p\) parcourt \(\mathbb{Z}\), est l'ensemble des nombres pairs.
\item La réunion d'une partie \(P\) d'un ensemble \(E\) et de son complémentaire dans \(E\) est égale à \(E\) tout entier :
  \[
    P\cup\overline{P}=E.
  \]
\end{enumerate}
Les propriétés suivantes sont immédiates :
\begin{enumerate}[label=\alph*)]
\item quel que soit l'ensemble \(E\),
  \[E\cup E=E;\]
\item quels que soient les ensembles \(E\) et \(F\),
  \[E\cup F=F\cup E;\]
\item quels que soient les ensembles \(E\), \(F\) et \(G\),
  \[(E\cup F)\cup G=E\cup (F\cup G).\]
\end{enumerate}
Les parenthèses étant désormais inutiles, nous noterons \(E\cup F\cup G\) la valeur commune des deux membres (Fig. \ref{fig:reunion3ensembles}).

\begin{figure}[ht]
  \centering
  \begin{tikzpicture}
    \draw[pattern=dots] (0,0) circle [x radius=2cm,y radius=4cm];
    \draw[pattern=dots] (0,0) circle [x radius=2cm,y radius=4cm,rotate=120];
    \draw[pattern=dots] (0,0) circle [x radius=2cm,y radius=4cm,rotate=240];
    \node[circle,inner sep=1pt,fill=white] at (0,4) {E};
    \node[circle,inner sep=1pt,fill=white] at (-3.4,2) {G};
    \node[circle,inner sep=1pt,fill=white] at (3.4,2) {F};
    \node[inner sep=3pt,fill=white] at (0,0) {\(E\cup F\cup G\)};
  \end{tikzpicture}
  \caption{\label{fig:reunion3ensembles}Réunion de trois ensembles.}
\end{figure}

La réunion et l'intersection sont en quelque sorte compatibles avec l'inclusion ; plus précisément, soient \(E\), \(F\) et \(G\) trois ensembles. Si \(E\) est inclus dans \(F\), alors
\begin{equation}
  \label{eqn:intersectioninclusion}
  (E\cap G)\subset(F\cap G)
\end{equation}
et
\begin{equation}
  (E\cup G)\subset(F\cup G).
\end{equation}
Établissons par exemple la relation \ref{eqn:intersectioninclusion}. Soit \(x\) un élément de \(E\cap G\), c'est à dire un élément commun à \(E\) et à \(G\). Comme \(E\) est contenu dans \(F\), \(x\) appartient à \(F\) et \(G\), ce qu'il fallait prouver.


  

\end{document}
