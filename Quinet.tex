\documentclass[10pt,parskip=half,chapterprefix=true]{scrbook}
\usepackage{amsmath}
\usepackage{fontspec}
\setmainfont{TeX Gyre Pagella}
\setsansfont{TeX Gyre Heros}
\usepackage[math-style=french]{unicode-math}
\setmathfont{TeX Gyre Pagella Math}
\usepackage[french]{babel}
\usepackage{microtype}

\title{Cours élémentaire de mathématiques supérieures}
\author{Jean Quinet}

\newlength{\largeurperso}

\begin{document}

\frontmatter

\maketitle
\tableofcontents

\mainmatter

\part{Algèbre}

\chapter{Les ensembles}

\section{Ensembles et relations}

Le langage des ensembles, introduit il y a plus d'un siècle par Greg
\textsc{Cantor}, a envahi toutes les branches des mathématiques ; il
est employé en France à tous les niveaux de l'enseignement, y compris
à l'école maternelle. Nous employons donc ce langage, non seulement
parce qu'il est commode, mais aussi et surtout parce qu'il est est
désormais impossible de l'ignorer.

Les notions d'\emph{ensemble} et de \emph{relation} sont premières,
c'est à dire qu'il n'est pas possible de les définir à partir d'autres
notions.

Intuitivement, une relation est une affirmation portant sur un
ensemble \(E\), qui est vérifiée ou non pour un ensemble donné \(E_0\)
; suivant le cas, on dit que \(E_0\) satisfait ou non à cette
relation.

\subsection*{Négation d'une relation}

Étant donnée une relation \(R\), on définit en logique une nouvelle
relation, appelée \emph{négation} de \(R\), et notée
\((\text{non }R)\), ou encore \(\neg R\). La négation de la négation
de \(R\) est la relation \(R\) elle-même.

On dit encore que les relations \(R\) et \(\neg R\) sont
contradictoires, c'est à dire que
\begin{itemize}
\item pour tout ensemble, l'une des deux relations est vraie,
\item mais pour aucun ensemble, elles ne sont vraies l'une et l'autre.
\end{itemize}

\medskip Par exemple, dans l'ensemble des entiers naturels, les
relations
\begin{itemize}
  \settowidth{\largeurperso}{\(\neg\)}
\item [] \hspace{\largeurperso}\(R\) : l'entier \(x\) est pair,
\item [] \(\neg R\) : l'entier \(x\) n'est pas pair (autrement dit,
  est impair)
\end{itemize}
sont contradictoires.

\subsection*{Conjonction, disjonction}

On appelle \emph{conjonction} de deux relations \(R_1\) et \(R_2\), et on note (\(R_1\) et \(R_2\)), ou \(R_1\wedge R_2\), la relation vraie si et seulement si \(R_1\) et \(R_2\) le sont. On définit enfin la \emph{disjonction} des relations \(R_1\) et \(R_2\) : c'est la relation, notée (\(R_1\) ou \(R_2\)), ou \(R_1\vee R_2\), vraie si et seulement si l'une au moins des deux relations \(R_1\) et \(R_2\) est vraie.

\subsection*{Implication}

À partir de ces trois définitions fondamentales, on introduit l'\emph{implication} : étant données deux relations \(R_1\) et \(R_2\), la relation (\(R_2\) ou (non \(R_1\)) se note \(R_1\Rightarrow R_2\), et se lit «~\(R_1\) implique \(R_2\)~». Si l'implication \(R_1\Rightarrow R_2\) est vraie, tout ensemble satisfaisant à \(R_1\) satisfait à \(R_2\).

La relation d'implication conduit à la notion d'\emph{équivalence} de deux relations : on dit que les relations \(R_1\) et \(R_2\) sont équivalentes, et on note \(R_1\Leftrightarrow R_2\), si chacune d'elles implique l'autre.

L'implication \(R_2\Rightarrow R_1\) est dite \emph{réciproque} de l'implication \(R_1\Rightarrow R_2\). L'implication \((\text{non}R_2)\Rightarrow(\text{non} R_1)\) est appelée \emph{contraposée} de \(R_1\Rightarrow R_2\). Une implication et sa contraposée sont toujours équivalentes.

\paragraph{Exemple.} Soient les relations


\end{document}
